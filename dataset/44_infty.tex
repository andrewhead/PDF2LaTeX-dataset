\documentclass[a4paper,12pt]{article}
\usepackage{latexsym}
\usepackage{amsmath}
\usepackage{amssymb}
\usepackage{graphicx}
\usepackage{wrapfig}
\pagestyle{plain}
\usepackage{fancybox}
\usepackage{bm}

\begin{document}

0.1 Estimation

As in Assumption 5.1 of , we assume standard regularity conditions, namely that

conditional on $T = t, D = d$, and $M = m, Y$ is arandom draw from that sub-

population defined in terms of $t, d, m \in \{1,\ 0\}$. Furthermore, the outcome in the

subpopulations required for the identification results of interest must have compact

support and a density that is bounded from above and below as well as continu-

ously differentiable. Denote by{\it N}the total sample size across both periods and all

treatment-mediator combinations and by $i \in \{1,\ N\}$ an index for the sampled

subject, such that $(Y_{i},\ D_{i},\ M_{i},\ T_{i})$ correspond to sample realizations of the random

variables $(Y,\ D,\ M,\ T)$ . The total, direct, and indirect effects may be estimated

using the sample analogy principle, which replaces population moments with sam-

ple moments. For instance, any conditional mediator probability given the treat-

ment, $\mathrm{P}\mathrm{r} (M=m|D\ =\ d)$ , is to be replaced by an estimate thereof in the sample,

$I\{M_{i}=m,D_{i}=d\}\Sigma_{i=1}^{N}^{\Sigma_{i=1}^{N}}I\{D_{i}=d\}$. A crucial step is the estimation of the quantile-quantile trans-

forms. The application of such quantile transformations dates at least back to ,

see also , , and for recent applications. First, it requires estimating the condi-

tional outcome distribution, $F_{Y_{t}|D=d,M=m}(y)$ , by the conditional empirical distribu-

tion $F_{Y_{t}|D=d,M=m}(y)\ovalbox{\tt\small REJECT} = \displaystyle \frac{1}{\Sigma_{i=1}^{n}I\{D_{i}=d,M_{i}=m,T_{i}=t\}}\sum_{i:D_{i}=d,M_{i}=m,T_{i}=t}I\{Y_{i}\ \leq\ y\}$. Second,

inverting the latter yields the empirical quantile function $F_{t}^{-1}|D=d,M=m(q)$ . The em-

pirical quantile-quantile transform is then obtained by
\begin{center}
$\hat{Q}_{dm}(y) = F$ˆ$Y$-11$|D =d,M=m($ ˆ$Y$0$|D =d,M=m(y))$
\end{center}
This permits estimating the average and quantile effects of interest. Average effects

are estimated by replacing any (conditional) expectations with the corresponding

sample averages in which the estimated quantile-quantile transforms enter as plug-
\end{document}
