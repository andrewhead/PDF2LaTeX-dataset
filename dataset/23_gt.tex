\documentclass[12pt]{article}
\setlength{\topmargin}{-.3in}
\setlength{\oddsidemargin}{0in}
\setlength{\textheight}{8.2in}
\setlength{\textwidth}{6.5in}
\setlength{\footnotesep}{\baselinestretch\baselineskip}
\newlength{\abstractwidth}
\setlength{\abstractwidth}{\textwidth}
\addtolength{\abstractwidth}{-6pc}
\usepackage{amsmath}
\usepackage{amsfonts}
\usepackage{amssymb}
\usepackage{latexsym}
\usepackage{epsf}
\usepackage{color}
\usepackage{graphicx}
\usepackage{tikz}
\usepackage{dsfont}
\usepackage{subfigure}
\usepackage{hyperref}
\pagestyle{plain}
\begin{document}

result given by
\begin{eqnarray}
Z = {1 \over |{\cal W}|} \int_{\mathrm{Cartan}}[d\sigma] \,\,e^{- {8 \pi^3 r \over g_{YM}^2} \mathrm{Tr}(\sigma^2)}\rm det \,_{\mathrm{Ad}}\left(\sin(i \pi \sigma) e^{{1\over 2} f(i \sigma)} \right)
\nonumber\\
\vphantom{.}\hspace{0.6 in} \times \prod_I \rm det \,_{R_I}\left((\cos(i \pi \sigma))^{1 \over 4} e^{-{1 \over 4} f({1\over 2} - i \sigma) - {1 \over 4} f({1\over 2} + i \sigma)} \right) + O\left(e^{- 16 \pi^3 r \over g_{YM}^2}\right)
\end{eqnarray}
where $r$ is the radius of $S^5$, $\sigma$ is a dimensionless matrix, and $f$ is defined as 
\begin{eqnarray}
f(y) = {i \pi y^3 \over 3} + y^2 \log\left(1-e^{-2 \pi i y} \right) + {i y \over \pi} \mathrm{Li}_2\left(e^{-2 \pi i y} \right) + {1 \over 2 \pi^2} \mathrm{Li}_3\left(e^{-2 \pi i y }\right) - {\zeta(3) \over 2 \pi^2}
\end{eqnarray}
The quotient by the Weyl group in  amounts to division by a simple numerical factor $|{\cal W}| = 2^N N!$. The integral over $\sigma$ is not restricted to a Weyl chamber. Though this localization result was obtained in the IR theory, it is expected to hold in the UV due to the assumed $Q$-exactness of the irrelevant UV completion terms. 
 One may rewrite the partition function in terms of the free energy as 
\begin{eqnarray}
Z = {1 \over |{\cal W}|} \int_{\mathrm{Cartan}}[d\sigma] \,e^{-F({\sigma})}+ O\left(e^{- 16 \pi^3 r \over g_{YM}^2}\right)
\nonumber\\
F(\sigma) = {4 \pi^3 r \over g_{YM}^2} \mathrm{Tr}\,\sigma^2 + \mathrm{Tr}_{\mathrm{Ad}} F_V(\sigma) + \sum_I \mathrm{Tr}_{R_I} F_H(\sigma)
\end{eqnarray}
The definitions of $F_V(\sigma)$ and $F_H(\sigma)$ follow simply from , and using  one may obtain the following large argument expansions
\begin{eqnarray}
F_V(\sigma) \approx {\pi \over 6} |\sigma|^3 - \pi |\sigma| \hspace{0.7 in} F_H(\sigma) \approx - {\pi \over 6} |\sigma|^3 - {\pi \over 8} | \sigma|
\end{eqnarray}
It was argued in  that in the large $N$ limit, the perturbative Yang-Mills term - i.e. the first term in the expression for $F(\sigma)$ in  -  can be neglected, as can be the instanton contributions. Thus in our evaluation of the free energy, we will only concern ourselves with the contributions coming from $F_V(\sigma)$ and $F_H(\sigma)$. 
The first step in the evaluation of  is recasting the matrix integral in a simpler form. The integral over $\sigma$ in  is an integration over the Coulomb branch, which is parameterized by the non-zero vevs of $\sigma$. One may write 
\begin{eqnarray}
\sigma = \mathrm{diag}\{\lambda_1,\dots, \lambda_N, - \lambda_1, \dots, -\lambda_N \}
\end{eqnarray}
since $USp(2N)$ has $N$ elements in its Cartan. The integration variables are these $N$ $\lambda_i$. Normalizing the weights of the fundamental representation of $USp(2N)$ to be $\pm e_i$ with $e_i$ forming a basis of unit vectors for $\mathbb R^N$, it follows that the adjoint representation has weights $\pm 2 e_i$ and $e_i \pm e_j$ for all $i \neq j$, whereas the anti-symmetric representation has only weights
\end{document}
