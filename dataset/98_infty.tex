\documentclass[a4paper,12pt]{article}
\usepackage{latexsym}
\usepackage{amsmath}
\usepackage{amssymb}
\usepackage{graphicx}
\usepackage{wrapfig}
\pagestyle{plain}
\usepackage{fancybox}
\usepackage{bm}

\begin{document}

such a splitting is referred to as {\it modular} In our Rényi-Ulam game variation,

the expected values of FDR and FOR are the error rates $e_{1}$ and $e_{0}$ in case the

truthful answer being yes and no, respectively. Fig. displays the error rates {\it e} $0$

and $e_{1}$ as functions of ?. For ?$= 31,$ we compute $e_{0} \approx 0.052$ and $e_{1} \approx 0.007,$

i.e. we have an error rate of0.052 for rejecting and an error rate of 0.007 for

confirming a base pair.

0.1. {\it Entropy}

To quantify the uncertainty of an ensemble, we define the {\it structural entropy}

of an ensemble, $\Omega$, of an RNA sequence, $\mathrm{x}$, as the Shannon entropy
$$
H(\Omega)=-\sum_{s\in\Omega}p(s)\log_{2}p(s)\ ,
$$
Proposition implies that a sample with small structural entropy contains a dis-

tinguished structure and a proof is given in . We refer to asample having a

distinguished structure of probability at least ? as being $\ovalbox{\tt\small REJECT}$-{\it distinguished}.

Next we quantify the reduction of a bit query on an ensemble. Recall that the

associated r.v. $X_{i,j}$ of a base pair $(i,\ j)$ partitions the sample $\Omega$ into two disjoint

sub-samples $\Omega_{0}$ and $\Omega_{1}$, where $\Omega_{k} = \{s\ \in\ \Omega\ :\ X_{i,j}(s)\ =\ k\} (k\ =\ 0,1)$ . The

{\it conditional entropy}, $H(\Omega|X_{i,j})$ , represents the expected value of the entropies

of the conditional distributions on $\Omega$, averaged over the conditioning r.v. $X_{i,j}$

and can be computed by
$$
H(\Omega|X_{i,j})=(1-p_{i,j})H(\Omega_{0})+p_{i},{}_{j}H(\Omega_{1})\ .
$$
Then the {\it entropy reduction} $R(\Omega,\ X_{i,j})$ of $X_{i,j}$ on $\Omega$ is the difference between

the {\it a priori} Shannon entropy $H(\Omega)$ and the conditional entropy $H(\Omega|X_{i,j})$ , i.e.
$$
R(\Omega,\ X_{i,j})=H(\Omega)-H(\Omega|X_{i,j})\ .
$$
$\mathbb{P}(s\in\Omega^{*})=(1-e_{0})^{l_{0}}(1-e_{1})^{l_{1}}$, where $l_{0}$ and $l_{1}$ denote the number of No-/Yes-

answers to queried base pairs along the path, respectively. Fig. displays the

distribution of $l_{1}$. We observe that {\it l}1 has amean around 5, i.e., the probabilities

of queried base pairs being confirmed and being rejected are roughly equal.

For $l_{0} = l_{1} = 5$, we have atheoretical estimate $\mathbb{P} (s\ \in\ \Omega^{*}) \approx 0.736$. In Fig.

we present that $\mathbb{P} (s\ \in\ \Omega^{*})$ decreases as the error rate $e_{0}$ increases, for fixed

$e_{1}=0.01.$

1. Information theory

As the Boltzmann ensemble is a particular type of discrete probability spaces,

the information-theoretic results on the ensemble trees will be stated in the more

general setup. Let $\Omega = (\mathrm{S},\ \mathrm{P}(\mathrm{S}),p)$ be a discrete probability space consisting

of the sample space $\mathrm{S}$, its power set $\mathrm{P}(\mathrm{S})$ as the s-algebra and the probability

measure {\it p}. The {\it Shannon entropy} of $\Omega$ is given by
$$
H(\Omega)=-\sum_{s\in \mathrm{S}}p(s)\log_{2}p(s)\ ,
$$
1
\end{document}
