\documentclass[a4paper,12pt]{article}
\usepackage{latexsym}
\usepackage{amsmath}
\usepackage{amssymb}
\usepackage{graphicx}
\usepackage{wrapfig}
\pagestyle{plain}
\usepackage{fancybox}
\usepackage{bm}

\begin{document}

with zero corresponding to a NA value. Next we want to identify the

bivariate outliers by iterating the filter over all possible pairs of variables.

Consider a pair of variables $\mathrm{X}^{(\mathrm{j}\mathrm{k})} = \{(\mathrm{X}_{\mathrm{i}\mathrm{j}},\ \mathrm{X}_{\mathrm{i}\mathrm{k}})\}, \mathrm{i} = 1$, . . . , $\mathrm{n}$. The ini-

tial location and dispersion estimators are, respectively, the coordinate-wise

median and the $2 \times 2$ sub-matrix $S^{(jk)}$ of the estimate $S$ computed by the

generalized S-estimator on non-filtered data. Note that, this ensure the pos-

itive definiteness property for $S$ and each $d\times d$ sub-matrix corresponding to

a subset of $d$ variables. For bivariate points with no flagged components by

the univariate filter we compute the squared Mahalanobis distance $\ovalbox{\tt\small REJECT}_{i}^{(jk)}$ and

hence apply the bivariate filter, for all $1 <j < k <p$. At the end we want

to identify the cells $(i,j)$ which have to be flagged as cell-wise outliers. The

procedure used for this purpose is described in ? and reported here. Let

$J=$\{$(i,j,\ k)$ : $\ovalbox{\tt\small REJECT}_{i}^{(jk)}$ is flagged as bivariate outlier\}

be the set of triplets which identifies the pairs of cells flagged by the bivariate

filter in rows $i=1$, . . . , $n$. For each cell $(i,\ j)$ in the data, we count the number

of flagged pairs in the {\it i}-th row in which the considered cell is involved:
$$
m_{ij}=\#\{k:\ (i,j,\ k)\in J\}.
$$
In absence of contamination, $m_{ij}$ follows approximately a binomial distribu-

tion {\it Bin} $(\displaystyle \sum_{k\neq j}\mathrm{U}_{\mathrm{j}\mathrm{k}}$, d$)$ where d represents the overall proportion of cell-wise

outliers undetected by the univariate filter. Hence, we flag the cell $(i,\ j)$ if

$m_{ij} > c_{ij}$ , where $c_{ij}$ is the 0.99-quantile of {\it Bin} $(\displaystyle \sum_{k\neq j}\mathrm{U}_{j\mathrm{k}},\ 0.1)$ . Finally, we

perform the {\it p}-variate filter as described in subsection to the full data matrix.

Detected observations (rows) are directly flagged as {\it p}-variate (case-wise) out-

liers. We denote the procedure based on univariate, bivariate and {\it p}-variate

filters, HS-UBPF.

0.1 A sequencing filtering procedure

Suppose we would like to apply a sequence of $k$ filters with different dimension

$1 \leq d_{1} \leq d_{2} \leq.$ . . $\leq d_{k} \leq p$. For each $d_{i}, i = 1$, . . . , $k$, the filter updates

the data matrix adding NA values to the {\it di}-tuples identified as {\it di}-variate

outliers. In this way, each filter applies only those {\it di}-tuples that have not

been flagged as outliers by the filters with lower dimension. Initial values for

each procedures rather than $d_{1}$ would be obtained by applying the GSE to

the actual filtered values. This procedure aims to be avalid alternative to

that used in the presented HS-UBPF filter to perform a sequence of filters

with different dimensions. However, this is a preliminary idea, indeed it has

not been implemented yet.

1
\end{document}
