\documentclass[12pt]{article}
\setlength{\topmargin}{-.3in}
\setlength{\oddsidemargin}{0in}
\setlength{\textheight}{8.2in}
\setlength{\textwidth}{6.5in}
\setlength{\footnotesep}{\baselinestretch\baselineskip}
\newlength{\abstractwidth}
\setlength{\abstractwidth}{\textwidth}
\addtolength{\abstractwidth}{-6pc}
\usepackage{amsmath}
\usepackage{amsfonts}
\usepackage{amssymb}
\usepackage{latexsym}
\usepackage{epsf}
\usepackage{color}
\usepackage{graphicx}
\usepackage{tikz}
\usepackage{dsfont}
\usepackage{subfigure}
\usepackage{hyperref}
\pagestyle{plain}
\begin{document}

\subsection{BPS Equations}
We now use the vanishing of the fermionic variations to obtain BPS equations for the warp factor and the three non-zero scalars. 
\subsubsection{Dilatino equation and projector}
We begin by imposing the vanishing of the dilatino variation, $\delta\chi_A = 0$, which implies
\begin{eqnarray}
{1\over 2} \gamma^5 \sigma' \varepsilon_A = N_0 \varepsilon_A + N_3 \gamma^7 {(\sigma^3)^B}_A \varepsilon_B
\end{eqnarray}
This equation can be interpreted as a projection condition on the spinors $\varepsilon_A$. Consistency of this projection condition then requires that 
\begin{eqnarray}
 \sigma' = 2 \eta \sqrt{N_0^2 + N_3^2}
\end{eqnarray}
where $\eta = \pm1$. Plugging this BPS equation back into then yields a second form of the projection condition,
\begin{eqnarray}
\gamma^5 \varepsilon_A = G_0 \varepsilon_A - G_3 \gamma^7 {(\sigma^3)^B}_A \varepsilon_B
\end{eqnarray}
which is more useful in the derivation of the other BPS equations. In the above, we have defined 
\begin{eqnarray}
G_0 = \eta {N_0 \over \sqrt{N_0^2 + N_3^2}}\hspace{1 in} G_3 = -\eta {N_3 \over \sqrt{N_0^2 + N_3^2}}
\end{eqnarray}
\subsubsection{Gravitino equation}
The analysis of the gravitino equation  $\delta \psi_{A \mu}=0$ proceeds in exactly the same way as for the Lorentzian case studied in . The procedure gives rise to a first-order equation for the warp factor $f$ and an algebraic constraint. To avoid excessive overlap with that paper, we simply cite the result,
\begin{eqnarray}
f^{\prime }=2(G_{0}S_{0}+G_{3}S_{3})\hspace {.5in}e^{-2f}=4(G_{0}S_{0}+G_{3}S_{3})^{2}-4(S_{0}^{2}+S_{3}^{2})\hspace {.5in}(5)
\end{eqnarray}
\subsubsection{Gaugino equations}
Finally, we turn toward the gaugino equation $\delta \lambda^I_A=0$. Again the analysis of this equation proceeds in an exactly analogous manner to the Lorentzian case . The result is
\begin{eqnarray}
\cos \phi^3 (\phi^0)' = - (G_0 M_0 + G_3 M_3) \hspace{0.5 in} (\phi^3)' = i (G_3 M_0 - G_0 M_3)
\end{eqnarray}
The right-hand sides of both equations are real, and thus give rise to real solutions when appropriate initial conditions are imposed.
\end{document}
