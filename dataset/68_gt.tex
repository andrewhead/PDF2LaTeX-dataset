\documentclass[conference]{IEEEtran}
\usepackage{cite}
\usepackage[pdftex]{graphicx}
\usepackage{amsmath,amssymb,amsfonts}
\usepackage{algorithmic}
  \usepackage[caption=false,font=footnotesize]{subfig}
\usepackage{url}
\usepackage{xcolor}
\usepackage{eurosym}

\begin{document}
 solar thermal on-site do not have $CO_2$ associated. Embedded emissions were not included.
For additional details on the model and the references of the input data used in the model, refer to .
\section{Grid Tariffs Description}
The Norwegian electricity consumption's recent trend is a consumption where peak demand increases relatively more than annual demand. This trend must be met by new incentives to shave peak load in order to avoid costly distribution grid investments. Grid tariffs are one effective way to solve this issue. In this paper we suggest three new grid tariffs and compare the results with the current grid tariff.
The first analyzed grid tariff is energy based and is the current tariff in Norway. It consists of an annual fixed price and a grid energy cost per kWh consumed. As this rate is flat, it does not incentivize flexible resources nor consumption patterns which results in lower peak demand. The annual cost can be calculated using \eqref{energytarifftotal}.
\begin{equation}
    C^{tot} = 137 + 0,0225\cdot \sum_t y_{t}^{imp\_tot}
\end{equation}
The second grid tariff is a time-of-use based tariff which penalizes import when there is typically scarcity in the grid. The tariff has a basic cost, which is double during peak load hours (7-10am and 6-9pm) and reduced to half during low load hours (11pm-5am). The effect of increasing electric vehicle and demand response penetration on the peak hours is ignored. The total costs are given by \eqref{TOUtotal}.
\begin{multline}
    C^{tot} = \sum_t \big( 0,0123\cdot y_{t}^{imp\_low} + 0,0246\cdot y_{t}^{imp\_med} \\ 
    + 0,0492\cdot y_{t}^{imp\_peak} \big)
\end{multline}
The third tariff was originally described in , and is called capacity subscription. It contains a fixed annual cost (\euro/year), a capacity cost (\euro/kW), an energy cost (\euro/kWh) and an excess demand charge (\euro/kWh). The energy cost is significantly higher when the imports are above the subscription. The main advantage of this tariff is that it incentivizes peak shaving and creates a market for capacity where consumers pay for the resource which in fact is scarce in the distribution grid: capacity. Disadvantages are complexity and the uncertainty in consumer behaviour. In addition, the optimal subscribed capacity is unknown in advance. Finding its value is further discussed in . In this paper, the subscribed capacity is a variable in the optimization. In reality, the consumer would have to choose it and it would most likely not be the optimal value.
The costs are calculated with \eqref{subcaptotal}.
\begin{equation}
   C^{tot} = 108 \cdot c^{sub} + \sum_t \big( 0,005\cdot y_{t}^{imp\_below} + 0,1\cdot y_{t}^{imp\_above} \big)
\end{equation}
The fourth tariff is a dynamic tariff where grid scarcity is taken into account. As an extra incentive to reduce impacts on the grid, a penalization $C^{sc}$ is given for consumption in hours with grid scarcity. Scarcity $\delta _{t}^{sc}$ in the system is defined as the 5\% of hours in the region (NO1) when the load is the highest. The percentage chosen is arbitrary and could be tuned or changed into a threshold by the regulator. The total costs are given by \eqref{dynamictotal}. In addition, as an added incentive to help the grid, a bonus for exporting in those hours is added, at the same cost as the scarcity tariff. In \eqref{dynamictotal}, $\delta _{t}^{sc}$ is a binary parameter defining for each hour if there is scarcity in the grid.
\begin{multline}
    C^{tot} = \sum_t \Big( \big( 0,0225\cdot(1-\delta _{t}^{sc}) + \delta _{t}^{sc} \cdot 0,1 \big) \cdot y_{t}^{imp\_tot} \\ - 0,1 \cdot  \delta _{t}^{sc} \cdot y_{t}^{exp\_tot} \Big)
\end{multline}
\section{Results}
In Norway, the legislation regarding prosumers is changing, moving from a situation where exports are limited to 100kW to a situation of unrestrained export. For this reason, both cases are investigated to explore the consequences on the design of ZENs of the different grid tariffs in these cases.
The investment in the energy system can be seen in Table  and in Table , respectively for the case without and with limitation on exports. The results are presented in the format Prod Plant/ Student Housing/ Normal Offices/ Passive Offices.
The investments stay similar, no new technology is introduced or replaced. However, small variations in the amount of each technology appear, in particular heat storage. The difference between the energy system with and without export limit is greater, namely due to storages. A large battery pack is necessary in order to store the PV production while it waits to be exported, i.e. to accommodate the bottleneck. In addition, large investments in heat storages and electric boilers are done.
The subscribed capacity resulting of the optimization is of 134,5kW for the case with no export limit, and of 124kW in the case with export limits.
Fig.  presents the total cost of the neighborhood's energy system (investment and operation) and the total revenue for the DSO, both over the lifetime and discounted to the start of the study. 
There are small variations in the cost in all cases. Subscribed capacity and dynamic pricing cause an increase in the total cost for the ZEN between 3 and 5\% compared with the energy case. On the other hand, the time of use scheme allows for a cost reduction of around 12\% in the case without export limit and 5\% with export limit.
The DSO revenue from the ZEN are higher when using the other pricing schemes than with the energy scheme when there is no export limit. When there is export limits, the DSO revenue stays the same because the battery allows to self-consume more and "anticipates" the higher price periods and buys electricity when the price is lower. The revenue in the case of export limits are about half of the revenue of the case of no export limit except in the case of subscribed capacity where the subscription tariff allows to maintain the revenue. 
The cost increase in the ZEN is of the same order of magnitude as the increase in revenue for the DSO except for ToU where the cost of the ZEN decreases while the revenue for the DSO increases. ToU has a beneficial effect from both points  of view in this aspect.
The duration curves Fig. , in the case of no export limit, are not affected much by the tariff scheme in place. 
\end{document}
