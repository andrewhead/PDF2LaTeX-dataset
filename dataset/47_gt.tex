\documentclass[a4paper,12pt]{article}
\usepackage[english]{babel}
\usepackage[utf8x]{inputenc}
\usepackage[T1]{fontenc}
\usepackage[flushmargin]{footmisc}
\usepackage{setspace}
\usepackage[comma]{natbib}
\usepackage{float}
\usepackage{amsmath}
\usepackage{amsfonts}
\usepackage{amssymb}
\usepackage{ae}
\usepackage{caption}
\usepackage[a4paper,top=3cm,bottom=2cm,left=3cm,right=3cm,marginparwidth=1.75cm]{geometry}
\usepackage{graphicx}
\usepackage[colorinlistoftodos]{todonotes}
\usepackage[colorlinks=true, allcolors=blue]{hyperref}

\begin{document} \doublespacing \pagestyle{plain}
of the respective estimators of $\theta_1^n$, $\theta_1^a$, $\Delta_c$,  $\theta_1^c(1)$, $\theta_1^c(0)$, $\delta_1^c(1)$, and $\delta_1^c(0)$ for the linear model. In this case, the identifying assumptions underlying both the change-in-changes (Panel A.) and difference-in-differences (Panel B.) estimators are satisfied. Specifically, the homogeneous time trend on the individual level satisfies any of the common trend assumptions in , while the monotonicity of $Y$ in $U$ and the independence of $T$ and $U$ satisfies the key assumptions of this paper. For this reason any of the estimates in Table  are close to being unbiased and appear to converge to the true effect at the parametric rate when comparing the results for the two different sample sizes.
Table  provides the results for the exponential outcome model, in which the time trend is heterogeneous and interacts with $U$ through the nonlinear link function. While the change-in-changes assumptions hold (Panel A.), average time trends are heterogeneous across complier types such that the difference-in-differences approach (Panel B.) of  is inconsistent. Accordingly, the biases of the change-in-changes estimates generally approach zero as the sample size increases, while this is not the case for the difference-in-differences estimates. Change-in-changes yields a lower root mean squared error than the respective difference-in-differences estimator in all but one case (namely $\hat{\delta}_1^c(0)$ with $N=1,000$) and its relative attractiveness increases in the sample size due to its lower bias.
In our final simulation design, we maintain the exponential outcome model but assume $D$ to be selective w.r.t.\ $U$ rather than random. To this end, the treatment model in is replaced by $D=I\{U+Q>0\}$, with the independent variable $Q\sim N(0,1)$ being an unobserved term. Under this violation of Assumption 7, complier shares and effects are no longer identified, which is confirmed by the simulation results presented in Table . The bias in the change-in-changes based total, direct, and indirect effects on compliers do not vanish as the sample size increases. Furthermore, under non-random assignment of $D$ (while maintaining monotonicity of $M$ in $D$), the never-takers' and always-takers' respective distributions of $U$ differ across treatment. Therefore, average direct effects among the total of never or always-takers, respectively, are not identified. Yet, $\theta_1^{1,0}$, which is still identified by the same estimator as before, yields the direct effect
\end{document}
